% !TEX options = -synctex=1 -interaction=nonstopmode -file-line-error "%DOC%"
% !TEX program = lualatex
\documentclass[]{article}
\usepackage{amsmath}
\usepackage{amsfonts}
\usepackage[english]{babel}
\usepackage{amsthm}
\usepackage{mathtools}
\usepackage{hyperref}
\usepackage[final]{graphicx}
\usepackage{listings}
% \usepackage{minted}

% Basic type settings ----------------------------------------------------------
\usepackage{courier}
\usepackage[margin=1in,footskip=0.25in]{geometry}
\linespread{1}                           % lin spacing
\usepackage[fontsize=12pt]{fontsize}

% Theorems, Lemma, Propositions etc --------------------------------------------
\theoremstyle{definition}
\newtheorem{theorem}{Theorem}            % Theorem counter global 
\newtheorem{prop}{Proposition}[section]  % proposition counter is section
\newtheorem{lemma}{Lemma}[subsection]    % lemma counter is subsection
\newtheorem{definition}{Definition}
\newtheorem{remark}{Remark}[subsection]

% Other settings ---------------------------------------------------------------
\hypersetup{
    colorlinks=true,
    linkcolor=blue,
    filecolor=magenta,
    urlcolor=cyan,
}
\lstset{basicstyle=\footnotesize\ttfamily,breaklines=true}
\newcommand{\indep}{\perp \!\!\! \perp}
\usepackage{wrapfig}
\graphicspath{{.}}
\usepackage{fancyvrb}


% Julia related stuff: https://github.com/mossr/julia-mono-listings ------------
% !!! You will need LuaLaTeX for this 
\input{julia_font}
\input{julia_listings}
\lstdefinelanguage{JuliaLocal}{
    language = Julia, % inherit Julia lang. to add keywords
    morekeywords = [3]{thompson_sampling}, % define more functions
    morekeywords = [2]{Beta, Distributions}, % define more types and modules
}

\begin{document}
\numberwithin{equation}{subsection}
    \LaTeX
    \begin{lstlisting}[language=JuliaLocal, style=julia]
include("util.jl")
include("../../src/CGPO.jl")  
        
function PerformCGFor(
    A::AbstractMatrix, 
    b::AbstractVecOrMat;
    epsilon=1e-2,
    exact::Bool=true, 
    partial_ortho=nothing,
)
    cg = CGPO(A, b)
    
    if exact
        
    else
        if partial_ortho === nothing
            cg |> TurnOffReorthgonalize!
        else
            StorageLimit!(cg, partial_ortho)
        end
    end

    ẋ = A\b
    ė = ẋ - cg.x
    ėAė = dot(ė, A*ė)
    E = Vector{Float64}()
    push!(E, 1)
    RelErr = 1
    while RelErr > epsilon
        cg()
        e = ẋ - cg.x
        RelErr = (dot(e, A*e)/ėAė)|>sqrt
        push!(E, RelErr)
    end
    return E
end

E = PerformCGFor(Diagonal(rand(10)), rand(10))

function PerformExperiment1()
    N = 256
    A = GetNastyPSDMatrix(N, 0.9)
    b = rand(N)
    A = convert(Matrix{Float16}, A)
    b = convert(Vector{Float16}, b)
    # TODO: Make the plot distinguishable without colors. 

    # ==========================================================================
    # The exact computations
    # ==========================================================================

    RelErr = PerformCGFor(A, b, epsilon=1e-3, exact=true)
    k = length(RelErr)
    fig1 = plot(
        log10.(RelErr), 
        label="Relative Energy (exact)", 
        legend=:bottomleft
    )

    # ==========================================================================
    # No-Orthogonalizations
    # ==========================================================================

    RelErr = PerformCGFor(A, b, exact=false, epsilon=1e-3)
    k = length(RelErr)
    plot!(
        fig1, 
        log10.(RelErr), 
        label="Relative Energy (floats)",
        linestyle=:dash
    )

    # ==========================================================================
    # Theoretical Bounds
    # ==========================================================================
    ErrorsBound = [TheoreticalErrorBound(A, idx) for idx in 1: k]
    plot!(
        fig1, 
        log10.(ErrorsBound), 
        label="Theoretical Bound (exact)",
        xlabel="iteration count", 
        ylabel="relative error energy norm.",
        linestyle=:dot
    )

    # ==========================================================================
    # Floating Points Partially Orthogonalized
    # ==========================================================================
    RelErr = PerformCGFor(A, b, exact=false, epsilon=1e-3, partial_ortho=div(N, 8))
    k = length(RelErr)
    plot!(
        fig1, 
        log10.(RelErr), 
        label="Relative Energy (partial)",
        legend=:bottomleft, 
        linestyle=:dashdot
    )

    display(fig1)
    SaveFigToCurrentScriptDir(fig1, "fig1.png")

    
return end

PerformExperiment1()
        
    \end{lstlisting}
\end{document}